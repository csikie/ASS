\chapter{Összegzés} % Conclusion
\label{ch:sum}
A szakdolgozatomban bemutattam egy beadandó kezelő rendszer megvalósítását C\# alapokon. Az alkalmazásban különválnak a tárgyfelelősök, a gyakorlatvezetők és a hallgatók. A tárgyfelelős a hozzájuk tartozó tantárgyakhoz csoportokat hozhat létre, melyekhez gyakorlatvezetőket rendelhet. A gyakorlatvezetők feladatot írhatnak ki a csoport számára, valamint engedélyezhetik a hallgatók csoportba való jelentkezését. Végül a hallgatók látják a kiírt feladatokat melyekre megoldásokat küldhetnek be, amit a gyakorlatvezetők értékelhetnek.

A \emph{.NET} keretrendszer lehetővé teszi, hogy az alkalmazás több platformon is futhasson, Windows-on, MacOS-en és Linux-on.
\section{Továbbfejlesztési lehetőségek}
\begin{description}
    \item[Automata tesztelő:] a beadott megoldásokat legyen lehetőség automatán tesztelni \emph{Docker} segítségével, ezáltal segítve az oktatók munkáját a javításban.
    \item[Szerkeszthető feladatok:] legyen lehetőség a kiírt feladatokat módosítani.
    \item[Vizsga mód:] ha be van kapcsolva egy feladatnál, akkor a feladat ideje alatt a csoporthoz tartozó megoldott feladatok nem látszanak a felületen.
    \item[Tantárgy szintű feladatok:] legyen lehetőség egy tantárgy összes csoportjához ugyanazt a feladatot kiírni. Ezáltal egy tantárgynál, ha zárthelyi dolgozatnak a kiírásá sokkal könnyebben kezelhető lenne.
    \item[Értesítések és e-mail küldés:] a felhasználók kapjanak értesítéseket a rendszerben, ha például egy új feladat került kiírásra egy csoportjukban. Ezáltal a hallgatók figyelmét nem kerüli el, ha például új feladat kerül kiírásra, vagy értékelték egy beadott munkájukat. 
\end{description}
