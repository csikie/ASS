\chapter{Felhasználói dokumentáció} % User guide
\label{ch:user}

\section{Nyelvválasztás és bejelentkezés}
Az oldalra érkezve a kezdőoldalt láthatjuk, ahol egy üdvözlő üzenet fogad minket. Majd lehetőségünk nyílik bejelentkezni [TODO:kép ref] a rendszerbe, vagy a rendszer által támogatott lokalizációt tudjuk kiválasztani (erre majd még a bejelentkezés után is lehetőségünk nyílik lásd később,~\hyperref[step:mindenkinek-elerheto-oldal]{\ref{step:mindenkinek-elerheto-oldal}:~``Mindenki számára elérhető oldalak''}) [TODO: kép ref].
\subsection{Bejelentkezés}
A rendszerbe bejelentkezni az INF-es felhasználónkkal tudunk. Ha a bejelentkezés sikertelen volt, azt a rendszer hibaüzenetekkel jelzi a számunkra [TODO:kép ref]. Amennyiben a bejelentkezés sikeres volt, a felhasználói csoportnak megfelelő kezdőoldalon találjuk magunkat [TODO: kép ref].
\subsection{Nyelvválasztás}
A rendszer kilistázza a támogatott lokalizációkat (jelenleg magyar és angol). Alapértelmezett beállítás a magyar. Ezt felültudjuk írni, ha valamelyik gombra rákattintunk. [TODO: kép ref]
\section{Felhasználói csoportok}
A felhasználók négy csoportba tartozhatnak:
\begin{compactitem}
    \item \hyperref[step:admin-role]{Rendszergazda}
    \item \hyperref[step:teacher-role]{Tárgyfelelős}
    \item \hyperref[step:instructor-role]{Gyakorlatvezető}
    \item \hyperref[step:student-role]{Hallgató}
\end{compactitem}
Egy felhasználó tartozhat több felhasználói csoportba. Ha egy felhasználó több felhasználói csoportba is tartozik, akkor a felület menüsorán megjelenik egy "Szerepkör váltás" lenyitható menü, ahol a felhasználóhoz rendelt csoportokat találjuk, a kiválasztott linkre kattintva a csoporthoz tartozó kezdőoldalra navigáljuk magunkat [TODO: kép ref]. A felhasználóhoz a felhasználói csoportokat a felhasználó létrehozásakor is megadhatjuk, valamint a létrehozást követően tudjuk módosítani.
\subsection{Rendszergazda}\label{step:admin-role}
A rendszergazda a következő funkciókat érheti el:
\begin{compactitem}
    \item Tantárgy létrehozása, módosítása, törlése, tárgyi információk megtekintése
    \item Felhasználó létrehozása, módosítása, a felhasználók adatainak a megtekintése
\end{compactitem}
Ha a rendszergazdaként jelentkezünk be az alábbi két táblázat fogad minket a kezdőoldalon~(\ref{fig:admin-page} ábra). 
\begin{figure}[H]
	\centering
	\subfigure[Tantárgyak táblázata]{
		\includegraphics[width=0.45\linewidth]{elte_cimer_szines}
        \label{subfig:admin-subject-table}}
	\hspace{5pt}
	\subfigure[Felhasználók táblázata]{
		\includegraphics[width=0.45\linewidth]{elte_cimer_szines}
        \label{subfig:admin-users-table}}
	\caption{Rendszergazdai felhasználó csoport kezdőoldala}
	\label{fig:admin-page}
\end{figure}
Az első táblázatban a rendszerben létrehozott tantárgyak és a hozzájuk tartozó információk olvashatóak le. A táblázatban az egyes tantárgyakhoz tartozó adatok módosíthatóak, illetve az egész tárgyat lehet törölni. A módositás során validálásra kerül, hogy a módosított név létezik-e már a rendszerben, ha igen, akkor ezt a rendszer jelzi számunkra. A második táblázatban a rendszerben létrehozott felhasználókat és a hozzájuk tartozó információkat láthatjuk. A rendszergazdának a felhasználók adatait és felhasználói csoport jogait tudja módosítani~(\ref{fig:admin-user-edit} ábra).
\begin{figure}[H]
	\centering
	\includegraphics[width=0.6\textwidth]{elte_cimer_szines}
	\caption{Felhasználó módosítása}
	\label{fig:admin-user-edit}
\end{figure}
\subsubsection{Tantárgy létrehozása}
A ``Tárgy létrehozás'' linkre kattintva az alkalmazás átnavigál minket egy űrlapra, ahol az új tantárgy szükséges adatait tudjuk kitölteni ([TODO: kép ref]). Ha az adatok validálása és feldolgozása sikeres, akkor vissza navigálódunk a kezdőoldalra. Az esetleges validalási hibákat a rendszer jelzi számukra ([TODO: kép ref]).
\subsubsection{Felhasználó létrehozása}
Felhasználót létrehozni a ``Felhasználó létrehozás'' linkre kattintva tudjuk megtenni, ami tovább navigál minket egy űrlapra, ahol az új felhasználónak az adatait tudjuk megadni ([TODO: kép ref]). Az adatok validálása sikeres, akkor a felhasználó elkészült és vissza navigálódunk a kezdőoldalra, ha nem volt sikeres, akkor a rendszer ezt hibaüzenetekkel jelzi nekünk ([TODO: kép ref]).
\subsection{Tárgyfelelős}\label{step:teacher-role} 
\subsection{Gyakorlatvezető}\label{step:instructor-role}
\subsection{Hallgató}\label{step:student-role}
\subsection{Mindenki számára elérhető oldalak}\label{step:mindenkinek-elerheto-oldal}
\section{Hibakezelés}